\documentclass[10pt,a4paper]{article}
\usepackage{hyperref}
\usepackage{listings}
\usepackage{booktabs}

\begin{document}

\section{Introduction}
\textsc{ProtoMol} is an ``object-oriented, component based, framework for molecular dynamics (MD) simulations''. Written in \textsc{C++}, \textsc{ProtoMol} can simulate classical mechanical systems with very high accuracy and efficiency. For a detailed review of \textsc{ProtoMol}, see Ref \cite{Matthey2004}. In the remainder of this manual, we first explain the \textsc{ProtoMol} installation process, then we detail the usage of \textsc{ProtoMol} and its add-ons. At last, we describe the processing of writing customized add-ons for future simulation studies.

\section{Obtaining \textsc{ProtoMol}}
The original \textsc{ProtoMol} is hosted at \textsc{SourceForge}. The download address for linux version is 
\url{http://sourceforge.net/projects/protomol/files/ProtoMol/Protomol\%203.3}

\textsc{ProtoMol} is an open-source and cross-platform software hosted at \textsc{SourceForge}. Considerable modifications have been added for simulations in this dissertation, so it is highly recommended to have the modified version to start with, which can be found at \textsc{github} \url{https://github.com/kuangchen/ProtoMol}. Any \textsc{git} tool should be able to clone this online repository into local copies. 

\section{Compiling \textsc{ProtoMol}}
To compile modified \textsc{ProtoMol}, a \textsc{C++} compiler compatible with \textsc{C++11} standard is needed. Modified \textsc{ProtoMol} also depends on two external library, \textsc{Lua} v5.2 and \textsc{HDF5} v1.8.10, which can be found in the package repository (for Linux system) or downloaded online (for Windows system) at the following URL,
\begin{itemize}
\item \textsc{Lua} \url{http://code.google.com/p/luaforwindows/}
\item \textsc{HDF5} \url{http://www.hdfgroup.org/HDF5/release/obtain5.html}
\end{itemize}

The compilation and installation process is automated by \textsc{CMake} tool. On Linux system, installation is accomplished by the following commands, 

\begin{lstlisting}[language=bash]
# Remove previous CMakeCache and CMakeFiles folder
rm -rf CMakeCache.txt CMakeFiles

# Turn on build_lapack switch and set the build_lapack_type to lapack
cmake -DBUILD_LAPACK=ON -DBUILD_LAPACK_TYPE=lapack 

# Tell cmake to generate Makefiles
cmake .

# Install ProtoMol into system folder
sudo make install
\end{lstlisting}
The basic usage of \textsc{CMake} is documented at \url{http://www.cmake.org/cmake/help/documentation.html}. To toggle other build switches (cluster computing, $\cdots$), see \textsc{CMakeLists.txt} for detail.

\section{Basic Usage}
To start a simulation a user needs to provide a configuration file (in plain text) that defines simulation initial conditions, integrator, outputs, and other properties of simulation. These entries are organized in the format of self-explanatory ``keyword - value'' pairs. The definition of integrators and forces have their own special syntax. A sample configuration file is presented below, with the meaning of each entry explained in the comment line. To see a full list of keywords, please see Ref \cite{Matthey2004, ProtoMolQuickRef}.
\begin{lstlisting}
# Number of steps in simulation
numsteps 100000000

# The number assigned to the first step
firststep 0

# Random Number seed
seed 11

# Initial temperature of particles
temperature 1e4

# Simulation cell size
cellsize 5000000

# Boundary Conditions
boundaryConditions vacuum

# Cell Manager
cellManager Cubic
exclude none

# Initial position and velocitiy definition
posfile ion_neutral_cooling_ini_pos_32.xyz
psffile ion_neutral_cooling_32.psf

# Par file definition
parfile ion_neutral_cooling.par

# Output Setting
outputfreq 10000

# Add IonSnapshot as the output
IonSnapshot ss.lua

# Integrator Setting
integrator {
  # 0th level integrator
  level 0 LeapfrogBufferGas {
    timestep 1e8
    filename buffer_gas.lua	
        
    # Add Coulomb force between ions
    force Coulomb 
    -algorithm NonbondedSimpleFull
        
    # Add ion trap force
    force LQT 
    -lqt_filename trap.lua
  }
}
\end{lstlisting}

\section{Using ProtoMol Add-Ons}
A set of new forces, integrators and outputs have been added specifically to simulate ion trap dynamics, whose usage are detailed below.

\subsection{New Forces}
\paragraph{Rf-trapping force} \mbox{} \\
Rf-trapping force is defined with the following syntax,
\begin{lstlisting}
force LQT 
    -lqt_filename your_file
\end{lstlisting}
where {\ttfamily{LQT}} is the name of the force, and the argument of {\ttfamily{-lqt\_filename}}, i.e. {\ttfamily{-your\_file}} is a \textsc{Lua} file that defines an ion trap. 

The trap definition declares a variable named {\ttfamily{trap}}, a \textsc{Lua} table that contains the mass of the ion in the trap, the physical dimensions of the trap and the trap voltage. A sample trap definition file is given below, where the meanings of each filed are the same as in Eq. \ref{eq:V} and their units are given in the comment.
\begin{lstlisting}
trap = { 
  m = 173,        -- AMU
  r0 = 12e-3,     -- m
  z0 = 21.5e-3,   -- m
  v_rf = 175/2,   -- V (Note arithmetic operations can be done in Lua file)
  v_ec = 10,      -- V
  eta = 0.1275,   -- 1
  omega = 300e3   -- Hz
}
\end{lstlisting}


\paragraph{Harmonic static trapping force}\mbox{} \\
The harmonic static trap is introduced to simulate ion dynamics under the secular approximation, in which ions are confined by a harmonic potential with the trapping frequency equal to ion's secular frequency. Harmonic trapping force is defined with the following syntax,
\begin{lstlisting}
force HarmonicTrapForce
    -ht_def your_file
\end{lstlisting}
where {\ttfamily{HarmonicTrapForce}} is the name of the force, and the argment of {\ttfamily{-ht\_filename}}, i.e. {\ttfamily{-your\_file}} is a \textsc{Lua} file that defines an harmonic trapping force. 

The trap definition declares a variable {\ttfamily{trap}}, a Lua table that contains a single field {\ttfamily{freq}}, which in turn contains three fields {\ttfamily{x, y, z}}. A sample trap definition file is given below, 
\begin{lstlisting}
trap = { 
  freq = { x = 39.63e3 * 2 * 3.14159,
           y = 39.63e3 * 2 * 3.14159,
           z = 39.63e3 * 2 * 3.14159 
         } 
}
\end{lstlisting}
The meanings and units of these fields are given in the following Tab. \ref{tab:harmonic_trap_def}.
\begin{table}[h!]
  \centering
  \begin{tabular}{ c  c  l } \toprule
    Field & Unit & Meaning\\ \midrule
    {\ttfamily{trap.freq.\{x,y,z\}}} & Hz & $\omega_{x,y,z}$ as defined in $V=\frac{1}{2}(\omega_x^2x^2+\omega_y^2y^2+\omega_z^2z^2)$\\\bottomrule
  \end{tabular}
  \caption{Harmonic trap definition syntax.}
  \label{tab:harmonic_trap_def}
\end{table}

\paragraph{Damping force}\mbox{}\\
The damping force, \textit{i.e.} $\vec{F} = -b \vec{v}$, is introduced to emulate the laser-cooling force, which quickly removes ions' kinetic energy, and cause the ions to form Coulomb crystal. Damping force is defined with the following syntax,
\begin{lstlisting}
force DampingForce
    -damping_def your_file
\end{lstlisting}
where {\ttfamily{DampingForce}} is the name of the force, and the argment of {\ttfamily{-damping\_def}}, i.e. {\ttfamily{-your\_file}} is a \textsc{Lua} file that defines a damping force. 

The trap definition declares a variable {\ttfamily{damping}} that contains the damping magnitude, and when the damping is on. A sample damping definition file is given below,
\begin{lstlisting}
damping = { 
  coeff = 3e-22,
  t_start = -1,
  t_end = 0.01 
}
\end{lstlisting}
The meanings and units of the fields in the {\ttfamily{damping}} table are given in Tab. \ref{tab:damping_def}.
\begin{table}[h!]
  \centering
  \begin{tabular}{ c  c  l } \toprule
    Field & Unit & Meaning\\ \midrule
    {\ttfamily{coeff}} & ? & damping coefficient $b$, defined by $\mathbf{F}=-b\mathbf{v}$\\
    {\ttfamily{t\_start}} & $s$ & start time. Laser is on only when {\ttfamily{t\_start}} $\le$ t $\le$ {\ttfamily{t\_{end}}}. \\
    {\ttfamily{t\_end}} & $s$ & end time  \\ \bottomrule
  \end{tabular}
  \caption{Damping force definition file}
  \label{tab:damping_def}
\end{table}

\subsection{New Integrators}
\paragraph{Buffer-gas Leapfrog  Integrator V2}\mbox{} \\
A leapfrog integrator is introduced to simulate random collisions with ultracold neutral atoms. It is defined with the following syntax,
\begin{lstlisting}
level X LeapFrogBufferGas2 {
  timestep 1e8
  filename your_file
}
\end{lstlisting}
where {\ttfamily{LeapFrogBufferGas2}} is the name of the integrator, and the argument of {\ttfamily{filename}}, i.e.{\ttfamily{your\_file}} is a \textsc{Lua} file that defines the buffer gas integrator, by declaring a table variable {\ttfamily{neutral}}. A sample buffer-gas leapfrog integrator definition file is given below
\begin{lstlisting}
neutral = { 
  mass = 40,
  polarizability = 50,
  temperature = 5e-3,
  density = 1e16 
}
\end{lstlisting}
The meanings and units of the fields in the {\ttfamily{neutral}} table are given in Tab. \ref{tab:leapfrogintegrator_def}
\begin{table}[h!]
  \centering
  \begin{tabular}{ c  c  l } \toprule
    Field & Unit & Meaning\\ \midrule
    {\ttfamily{mass}} & AMU & Mass of neutral atom \\
    {\ttfamily{polarizability}} & $10^{24} \mathrm{cm}^3$ & Polarizability of neutral atom \\
    {\ttfamily{temperature}} & K & Temperature of neutral atom \\
    {\ttfamily{density}} & $\mathrm{m}^3$ & Density of neutral atom \\\bottomrule
  \end{tabular}
  \caption{Buffer-gas leapfrog integrator definition file}
  \label{tab:leapfrogintegrator_def}
\end{table}

\subsection{New output}
\paragraph{Ion Snapshot}\mbox{}\\
At each simulation time step, ions' positions and velocities form a \textit{frame}. A consecutive series of \textit{frames} form a \textit{snapshot}. For analysis at a later time, an end user might want to record an array of snapshots starting at different time $t_i$ $(i=1,2,3,\cdots)$ during the simulation, and for simplify, let's suppose every snapshot contains the same number of frames. It is defined with the following syntax,
\begin{lstlisting}
  IonSnapshot your_file\end{lstlisting}
where {\ttfamily{your\_file}} is a \textsc{Lua} file that defines the ion snapshots.

The {\ttfamily{IonSnapshot}} definition declares a table variable {\ttfamily{ss}} that contains the number of snapshots, and frames in each snapshot. A sample {\ttfamily{IonSnapshot}} definition is given below, 
\begin{lstlisting}{Ion Snapshot syntax}
start_time = {}

for i=1, 50 do
   start_time[i] = (i-1) * 2e-2
end

ss = {
  num_frame = 250, 
  start = start_time 
  dir = ./
}
\end{lstlisting}
The meanings and units of the fields in the {\ttfamily{ss}} table are given in Tab. \ref{tab:ion_snapshot_def}
\begin{table}[h!]
  \centering
  \begin{tabular}{l  l  l  p{7cm} } \toprule
    Field & Type &  Unit & Meaning\\ \midrule
    {\ttfamily{num\_frame}} & integer & 1 & $N_f$, if set to -1, then $N_f$ is set to number of steps in the longest secular periods in $x,y,z$ three directions \\ 
    {\ttfamily{start}} & table & sec & $t_i$, i.e. the start time of each snapshot \\
    {\ttfamily{dir}} & string & N/A & The name of directory where snapshot files are stored\\
    \bottomrule
  \end{tabular}
  \caption{Snapshot Definition}
  \label{tab:ion_snapshot_def}
\end{table}

All the snapshot files are stored under the same directory, specified by the {\ttfamily{dir}} keywords in the definition file. They are named by the pattern snapshot\_{\ttfamily{i}}.hd5, where $i$ ranges from $0$ to $N_s-1$.
% \tikzstyle{every node}=[draw=black,thick,anchor=west]
% \tikzstyle{selected}=[draw=red,fill=red!30]
% \tikzstyle{optional}=[dashed,fill=gray!50]
% \begin{figure}[h!]
%   \centering
%   \begin{tikzpicture}[%
%     grow via three points={one child at (0.5,-0.7) and
%       two children at (0.5,-0.7) and (0.5,-1.4)},
%     edge from parent path={(\tikzparentnode.south) |- (\tikzchildnode.west)}]
%     \node {dir}
%     child { node {snapshot\_$0$.hd5} }
%     child { node {snapshot\_$1$.hd5} }
%     child { node {...} }
%     child { node {snapshot\_$(N_s-1)$.hd5} };
%   \end{tikzpicture}
%   \caption{Directory structure of snapshot output.}
%   \label{tkp:dir_structure}
% \end{figure}

The snapshot files are stored in HDF5 format, for its compact size and support by major programming languages, including C/C++/Python/Matlab. For an overview of HDF5 file, visit its official website at \url{http://www.hdfgroup.org/HDF5/}. Shown in Fig. \ref{tkp:ion_snapshot_file_struct}, every HDF5 file has a hierachical data structure, similar to the file system in an operating system. In addition to $N_f$ consecutive frames, a {\ttfamily{config}} header is also included which consists of auxillary information intended to make the snapshot file self-inclusive. The meanings of various fields in {\ttfamily{config}} is explained in Fig. \ref{tab:ion_snapshot_def2}.
% \tikzstyle{every node}=[draw=black,thick,anchor=west]
% \tikzstyle{selected}=[draw=red,fill=red!30]
% \tikzstyle{optional}=[dashed,fill=gray!50]
% \begin{figure}[h!]
%   \centering
%   \begin{tikzpicture}[%
%     grow via three points={one child at (0.5,-0.7) and
%       two children at (0.5,-0.7) and (0.5,-1.4)},
%     edge from parent path={(\tikzparentnode.south) |- (\tikzchildnode.west)}]
%     \node {/}
%     child { node {config} 
%       child { node {start} }
%       child { node {fps} }
%       child { node {numFrame} }
%       child { node {numAtom} }
%       child { node {numFrameperMM} }
%     }
%     child [missing] {}				
%     child [missing] {}				
%     child [missing] {}				
%     child [missing] {}				
%     child [missing] {}	
%     child { node {frame0} 
%       child {node {$x_1, y_1, z_1, v_{x,1}, v_{y,1}, v_{z,1}$} } 
%       child {node {$x_2, y_2, z_2, v_{x,2}, v_{y,2}, v_{z,2}$} } 
%       child {node {...} } 
%       child {node {$x_n, y_n, z_n, v_{x,n}, v_{y,n}, v_{z,n}$} }
%     }
%     child [missing] {}				
%     child [missing] {}				
%     child [missing] {}				
%     child [missing] {}			
%     child { node {frame1} 
%       child {node {$x_1, y_1, z_1, v_{x,1}, v_{y,1}, v_{z,1}$} } 
%       child {node {$x_2, y_2, z_2, v_{x,2}, v_{y,2}, v_{z,2}$} } 
%       child {node {...} } 
%       child {node {$x_n, y_n, z_n, v_{x,n}, v_{y,n}, v_{z,n}$} }
%     }
%     child [missing] {}				
%     child [missing] {}				
%     child [missing] {}				
%     child [missing] {}			
%     child { node {frame2} }
%     child { node {...} }
%     child { node {frameN} };
%   \end{tikzpicture}
%   \caption{File structure of each snapshot file}
%   \label{tkp:ion_snapshot_file_struct}
% \end{figure}

\begin{table}[h!]
  \centering
  \begin{tabular}{l  l  l  p{7cm} } \toprule
    Field & Type &  Unit & Meaning\\ \midrule
    {\ttfamily{start}} & float & s & Start time of the snapshot\\
    {\ttfamily{fps}} & integer & 1 & Number of frames per second \\
    {\ttfamily{numFrames}} & integer & 1 & Number of frames in the snapshot\\
    {\ttfamily{numAtoms}} & integer & 1 & Number of atoms in the simulation\\
    {\ttfamily{numFrameperMM}} & integer & 1 & Number of frames in one micromotion period\\
    \bottomrule
  \end{tabular}
  \caption{Snapshot config header}
  \label{tab:ion_snapshot_def2}
\end{table}

\section{Writing \textsc{ProtoMol} Add-Ons}
\textsc{ProtoMol} makes use of object-oriented programming language, and it has been carefully designed to ensure extensibility for new model encapsulation. How to add new modules to \textsc{ProtoMol} is beyond the scope of this thesis. For interested readers, it is recommended to read Ref. \cite{ProtoMolHowTo}, combined with the newest \textsc{ProtoMol} source code to 

\bibliography{ref.bib}
\end{document}